
\section{Espacios de Hilbert}

\textcolor{blue}{yo definir\'ia primero espacios de Banach.}

\begin{mydef}
Un producto interno $\langle\cdot,\cdot\rangle$ sobre un espacio vectorial
{\bf real} $X$ es una aplicaci\'on 
$\langle\cdot,\cdot\rangle:X\times X\to\mathbb{R}$,
que es lineal con respecto al primer y al segundo argumento
y es tal que 
$\langle\mathbf{x}, \mathbf{x}\rangle\geq0$ para todo $\mathbf{x}\in X$
y $\langle\mathbf{x}, \mathbf{x}\rangle=0$ si y s\'olo si $\mathbf{x}=0$.
\end{mydef}

\begin{mydef}
Un espacio de Hilbert $H$ sobre $\mathbb{R}$ es un espacio vectorial real
equipado con un producto interno $\langle\cdot,\cdot\rangle$, tal que
es completo con respecto a la norma
$\|\mathbf{x}\|^{2}= \langle\mathbf{x},\mathbf{x}\rangle$,
$\mathbf{x}\in H$.
\end{mydef}

Si bien todo espacio de Hilbert es un espacio de Banach, 
no todo espacio de Banach es un espacio de Hilbert.

\begin{mydef}
Un espacio de Banach $(X,\|\cdot\|)$ es un espacio de Hilbert
si y s\'olo si la norma $\|\cdot\|$ del espacio satisface la
identidad del paralel\'ogramo:
$$
\|\mathbf{x}+\mathbf{y}\|^2+\|\mathbf{x}-\mathbf{y}\|^2
= 2\|\mathbf{x}\|^2+2\|\mathbf{y}\|^2\quad
\text{para todo}\quad \mathbf{x},\mathbf{y}\in X\,.
$$
\end{mydef}

\begin{mydef}
Un espacio pre-Hilbert es un espacio vectorial equipado con un producto
interno o escalar.
Por eso, los espacios pre-Hilbert se denominan tambi\'en espacios con
producto interno.
No se exige que los espacios pre-Hilbert sean {\bf completos}.
Si adem\'as son completos, entonces se trata de espacios de Hilbert.
\end{mydef}

%Un espacio de Hilbert $H$ es un espacio vectorial equipado con un producto interno o producto escalar $\langle \cdot,\cdot \rangle$:

%\begin{equation}
%(H,\langle \cdot,\cdot \rangle) \Longrightarrow \text{ completo c/r  %}||\mathbf{x}||^2= \langle \mathbf{x},\mathbf{x} \rangle
%\end{equation}


%\begin{myteo}[de representación de Riesz]
%Si $H$ es un espacio de Hilbert y $\varphi$ un funcional %lineal y continuo:
%\begin{equation}
%\varphi: H \rightarrow \mathbb{C} \Longrightarrow \exists! %y_{\varphi} \in H \qquad \forall x \in H: \qquad \varphi(x) %= \langle x,y_{\varphi}\rangle 
%\end{equation}
%$y_{\varphi}$ se le denomina el representante.
%\end{myteo}


\begin{myteo}[de representaci\'on de Riesz]
Si $H$ es un espacio de Hilbert sobre un cuerpo escalar $\mathbb{K}$
($\mathbb{K}$ puede ser $\mathbb{R}$ o $\mathbb{C}$ en este curso)
y \ $\varphi: H \rightarrow \mathbb{K}$ un funcional lineal y continuo,
entonces existe un \'unico $y_{\varphi}\in H$ tal que
$\varphi(x)= \langle \mathbf{x},\mathbf{y}_{\varphi}\rangle$ para todo
$\mathbf{x}\in H$.
\end{myteo}

