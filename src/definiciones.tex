\bigskip
\section{Definiciones}\label{sec:def}

\begin{mydef}
Sea $X$ un $\mathbb{K}$-espacio vectorial
{\rm ($\mathbb{K}=\mathbb{C}$ o $\mathbb{R}$)}. 
Una \textbf{forma Hermitiana} positiva definida (p.d.) sobre $X$
es una aplicaci\'on:
\begin{eqnarray*}
\langle\cdot,\cdot\rangle: X\times X & \rightarrow & \mathbb{K} \\
(x,y) & \rightarrow & \langle x,y \rangle\,,
\end{eqnarray*}
que posee las siguientes propiedades:
\begin{enumerate}
\item
es aditiva con respecto al primer argumento:
$$
\langle x+x',y \rangle = \langle x,y \rangle + \langle x',y \rangle
\quad \forall x,x',y \in X\,;
$$
\item
la propiedad de homotecia:
$\langle\alpha x,y\rangle = \alpha\langle x,y\rangle$ 
para todo $\alpha\in\mathbb{K}$\,.
\item
la propiedad de anticonmutatividad:
$\langle x,y\rangle=\overline{\langle y,x\rangle}$
para todo $x,y\in X$
\end{enumerate}
\end{mydef}

\smallskip\noindent
De la definici\'on se desprende inmediatamente que la forma Hermitiana
$\langle\cdot,\cdot\rangle$ posee las siguientes propiedades:
\begin{enumerate}
\item
Es aditiva con respecto al segundo argumento:
$\langle x,y+y'\rangle = \langle x,y+y'\rangle+\langle x,y+y\rangle$.
\item
Es sesquilineal {\color{red}\bf ??? ?`Qu\'e sigue? ???}
\item
{\color{red}\bf Empty item?}
\end{enumerate}

\smallskip\noindent
Para $\mathbb{K}=\mathbb{R}$ las formas hermitianas son simplemente
formas bilineales sim\'etricas: 
$\langle x,y \rangle= \langle y,x \rangle$.
Adem\'as:
\begin{itemize}
\item
$\langle\cdot,\cdot\rangle$ es \textbf{positiva definida} ssi
$\langle x,x\rangle > 0\quad\forall x\in X$ con $x\neq0$.
\item
$\langle\cdot,\cdot\rangle$ es \textbf{positiva semidefinida} ssi
$\langle x,x\rangle\geq0\quad\forall x\in X$
(i.e., se admite la posibilidad que exista $u\in X$, $u\neq0$,
tal que $\langle u,u\rangle=0$).
\item
$\langle\cdot,\cdot \rangle$  es 
\textbf{un producto interno o producto punto} ssi 
$\langle\cdot,\cdot\rangle$ es \textbf{positiva definida}.
\end{itemize}

\medskip\noindent
\textbf{Ejemplo.}
Determinar un producto interno, distinto del producto punto can\'onico.

\smallskip\noindent
\emph{Desarrollo.}
Sea $X=\mathbb{R}^2$, $A\in{\rm Mat}(2\times 2,\mathbb{R})$, y definamos:
\begin{equation*}
\varphi: \mathbb{R}^2 \rightarrow \mathbb{R}\,,\qquad
(x,y)\mapsto \varphi(x,y):= x^{T}Ay\,,\quad x,y\in\mathbb{R}^2\,.
\end{equation*}
Entonces se tiene:
\begin{align*}
\varphi\quad\text{es positiva definida}\quad
& \Leftrightarrow\quad \varphi(x,y)> 0\quad
  \forall x,y\in\mathbb{R}^2\setminus\{0\} \\
& \Leftrightarrow\quad x^{T}Ay> 0\quad
  \forall x,y\in\mathbb{R}^2\setminus\{0\}\quad \\
& \Leftrightarrow\quad 
  A=\begin{bmatrix} a & b \\ c & d \end{bmatrix}\quad
  \text{es positiva definida} \\
& \Leftrightarrow\quad 
  a>0\quad\text{y}\quad ad-bc>0\,.
\end{align*}
En general, una matriz $A\in{\rm Mat}(n\times n,\mathbb{R})$
es positiva definida (p.d.) ssi sus subdeterminantes
{\color{red}\bf principales} son positivos definidos.
Por ejemplo, las matrices:
$$
I=\begin{bmatrix} 1 & 0 \\ 0 & 1 \end{bmatrix}
\quad\text{y}\quad 
A=\begin{bmatrix} 1 & 3 \\ 0 & 2 \end{bmatrix}
$$
son p.d.
Con ambas matrices se puede definir, entonces, un producto interno.
Por ejemplo, recurriendo a la matriz $A$ se obtiene el producto interno
sobre $\mathbb{R}^2$ dado por:
$$
\varphi(x,y) = \langle x,y\rangle := x^{T}Ay
= \begin{bmatrix} x_1 & x_2 \end{bmatrix}
  \begin{bmatrix} 1 & 3 \\ 0 & 2 \end{bmatrix}
  \begin{bmatrix} y_1 \\ y_2 \end{bmatrix}
%% = \begin{bmatrix} x_1 & x_2 \end{bmatrix} 
%%   \begin{bmatrix} y_1 + 3y_2 \\ 2 y_2 \end{bmatrix}
= x_1y_1 + 3x_1 y_2 + 2 x_2 y_2\,.
$$
Las lectoras verificar\'an que esta forma satisface los axiomas de una
forma bilineal p.d. o producto interno, en particular que:
$$
\|x\|_\varphi^2 := \varphi(x,x) = \langle x,x\rangle
= x_1^2 + 3x_1 x_2 + 2 x_2^2\,,\quad (x_1,x_2)\in\mathbb{R}^2\,,
$$
es efectivamente una norma sobre $\mathbb{R}^2$.


\begin{myremark}
En los p\'arrafos precedentes correg\'\i\ varios errores.
Sugiero pensar m\'as cuidadosamente los detalles de los textos
matem\'aticos que escriban.
Es muy f\'acil cometer errores si no se presta suficiente atenci\'on
a los detalles.

Omit\'\i\ el p\'arrafo que sigue, porque ya est\'a tratado en otra
parte de este documento
 
{\bf Definici\'on.}\quad{\em
Un funcional de evaluaci\'on sobre el espacio de funciones $H$ es
un funcional lineal $L_x: H \rightarrow \mathbb{R}$ que eval\'ua
cada funci\'on en $H$ en el punto $x$:
\begin{eqnarray*}
L_x: H &\rightarrow &\mathbb{R} \\
 f &\rightarrow & L_x(f):= f(x)
\end{eqnarray*}}

\end{myremark}
