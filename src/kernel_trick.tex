
\section{Kernel Trick}

\begin{mydef}
Dado un algoritmo formulado en t\'erminos de un kernel positivo
definido $k$, se puede construir un algoritmo alternativo
reemplazando $k$ por otro kernel positivo definido $\tilde{k}$.
\end{mydef}

\begin{myremark}
S\'olo puedo decit !`PLOP!
Aqu\'\i\ falta mucho trabajo todav\'\i a para que esta secci\'on
tenga sentido siquiera.
\end{myremark}

\subsection{Ejemplos de kernel}

\begin{description}
\item[Kernel Lineal]
$K(x,y) = \langle x,y \rangle$
\begin{myremark}
Aqu\'\i\ hay un problema: 
$x,y\in X$.
$X$ no tiene estructura de espacio de Hilbert, en general.
Luego, $\langle x,y\rangle$ est\'a indefinido.
\end{myremark}

\item[Kernel Gaussiano]
$K(x,y) = e^{-\frac{||x-y||^2}{\sigma^2}}, \sigma > 0$
\begin{myremark}
Aqu\'\i\ ocurre el mismo problema precedente: 
se usa una norma $\|\cdot\|$ sobre $X$, en circunstancias que $X$
no tiene, a priori, ninguna estructura.
\end{myremark}

\item[Kernel polinomial] 
$K(x,y) = \big( \langle x,y\rangle + 1\big)^d$, $d\in\mathbb{N}$
\begin{myremark}
Ocurre el mismo problema ya se\~nalado: ahora se requiere una estructura
de algebra sobre $X$.
\end{myremark}
\end{description}

\begin{myremark}
Los ejemplos de RKHS deben ser explicados mucho m\'as profundamente.
Tal como est\'an no dicen nada.
Aqu\'\i\ falta mucho trabajo todav\'\i a.
\end{myremark}
