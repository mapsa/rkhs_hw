\section{Idea Intuitiva de Kernels}
Supondremos que tenemos un conjunto con dos clases de objetos.
Al llegar un nuevo objeto, interesa saber a cu\'al de
las dos clases pertenece.
Este tipo de problema se denomina
{\bf problema de clasificaci\'on de objetos}. \cite{smola}.

Para poder realizar dicha clasificaci\'on en el contexto de m\'aquinas de
aprendizaje es necesario disponer de {\bf datos de entrenamiento}:

$$
(x_1,y_1),\,(x_2,y_2),\dots\,,\,(x_n,y_n)\ \in X\times\{-1,1\},
$$

\noindent donde $X$ es simplemente --por el momento-- un conjunto no vac\'\i o
y $x\in X$ corresponde al dato de entrada (tambi\'en denominados casos,
observaciones, patrones, etc.) e $y$ corresponde a la salida o etiqueta
asociada al dato de entrada.


Se espera que los algoritmos de m\'aquinas de  aprendizaje sean capaces de \emph{generalizar}, es decir,
los algoritmos deben predecir la etiqueta $y \in \{-1,1\}$
de los nuevos datos de entrada $x \in X$  mediante una funci\'on --o modelo-- constru\'ida
a partir de los datos de entrenamiento.
Para entender la idea, analicemos un ejemplo simple.
Supongamos que $X=\mathbb{R}$  y que los valores los valores de $x\in X=\mathbb{R}$ que son
mayores que $\pi$, por ejemplo, est\'an en la clase $y=+1$, y aquellos que son menores que $\pi$, est\'an en la clase $y=-1$.
Cuando el sistema recibe una entrada $x=\sqrt{11}$, \'este generaliza y responde que esta entrada pertenece a la clase $y=+1$.



La predicci\'on de la etiqueta $y\in\{-1,1\}$, debe ser de alguna manera
\textbf{similar} a los datos de entrenamiento $(x,y)$.
%
Para medir la similitud entre datos, se pueden usar diversas funciones.
En el caso de las etiquetas $y\in\{-1,1\}$, caracterizar la similitud es
simple, ya que s\'olo pueden ocurrir dos situaciones:
o bien las etiquetas son id\'enticas, por lo tanto similares,  o  bien son diferentes.
%Con esto queremos decir, por el momento en t\'erminos muy generales, que
%se escoge una etiqueta $y$ tal que $(x,y)$ es de alguna manera similar
%(parecido, cercano) a los datos de entrenamiento.



En el caso de los datos de entrada, la caracterizaci\'on de  su
similitud es un poco m\'as compleja, y es un aspecto central en el campo de
las llamadas m\'aquinas de aprendizaje.
Para poder definir la similitud entre $x,x'\in X$, comenzemos introduciendo una medida de similitud de la forma:
\begin{equation}
\begin{array}{rcl}
k: X\times X & \rightarrow & \mathbb{R} \\ 
    (x,x')   & \rightarrow & k(x,x')\;, \\
\end{array}
\end{equation}
donde $k$ es una funci\'on que recibe dos
par\'ametros, $x$ y $x'$, ambos pertenecientes al conjunto de entradas
$X$ y cuya salida es un valor real que caracteriza la similitud de ambos
par\'ametros.
Supondremos que $k$ es una funci\'on sim\'etrica,
es decir, $k(x,x')= k(x',x)$ para todo $x,x'\in X$.
Esta funci\'on $k$, {\em ser\'a\/} denominada {\bf kernel}.
Para comprender la forma general de los kernels, es necesario que
entendamos ciertos conceptos mat\'ematicos que estudiaremos en las siguientes
secciones,
as\'\i\ es que comenzaremos con un caso particular para llegar a entender su
forma general.


Una medida de similitud simple se podr\'ia definir mediante un \textbf{producto interno}
\footnote{Revisar en Sec. \ref{sec:def} la definici\'on de producto
interno a partir de formas hermitianas.}
(tambi\'en conocido como producto escalar o producto punto).

Por ejemplo, para los vectores $x,x'\in\mathbb{R}^n$, se define
el \textbf{producto interno can\'onico} de la siguiente manera:
\begin{equation}
\langle\mathbf{x},\mathbf{x'}\rangle
:= \sum \limits_{i=1}^{N} [\mathbf{x}]_i [\mathbf{x'}]_i 
\end{equation}
{\color{red} donde $[x]_i$, resp.  $[x']_i$, denota la $i$-\'esima
componente del vector $x\in\mathbb{R}^N$, resp.  $x'\in\mathbb{R}^N$.
Otro ejemplo de producto interno importante es:
\begin{equation*}
\langle f,g\rangle = \int_0^\infty f(t)\,\overline{g(t)}\;dt\,,\qquad
f,g\in C([0,1],\mathbb{C})\,,
\end{equation*}
donde $C([0,1],\mathbb{C})$ denota la clase de las funciones
continuas sobre el intervalo compacto $[0,1]$ con valores en
$\mathbb{C}$.
Obs\'ervese que $|\langle f,g\rangle|<\infty$ para todo
$f,g\in C([0,1],\mathbb{C})$.
{\bf ?`Por qu\'e?}}


\begin{myremark}
Aqu\'\i\ convendr\'\i a anotar los axiomas que debe satisfacer todo
producto interno.
{\bf Nota:}\ Estos axiomas se mencionan m\'as adelante en el texto.
\end{myremark}

\textcolor{blue}{Todav\'ia me falta llegar a decir: ``diremos que dos entradas
$x,x'\in X$ son similares cuando ...''}



